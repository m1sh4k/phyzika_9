\documentclass{article}
\usepackage{DejaVuSansCondensed} % шрифты документа - DejaVuSansCondensed
\renewcommand*\familydefault{\sfdefault}
\usepackage{graphicx}
%\usepackage{blindtext}
\usepackage{pgfplots}
\usepackage{cmap} % Таблицы кодировок. Корректная работа с кириллицей (поиск, копирование)
\usepackage[T1,T2A]{fontenc} % кодовая страница
\usepackage[utf8]{inputenc} % кодировка входного файла
\usepackage[english,russian]{babel}
\usepackage[linktocpage=true]{hyperref} % Чтобы работали ссылки, pdf metadata
\usepackage{hyphenat}
\hyphenpenalty=9000 % чем выше значение (до 10000), тем меньше переносов, default=50
\uchyph=1 % не переносить слова с большой буквы (не будут переноситься имена собственные), default=1
\tolerance=5000 % точность выравнивания правого края
\usepackage{setspace}
%\singlespacing     % 1
\onehalfspacing    % 1.5
%\setstretch{1.25}  % 1.5
%\doublespacing      % 2
%\linespread{1.25}
\def\paragraphindent{12.5mm}
\setlength{\parindent}{\paragraphindent} % абзацный отступ
\setlength{\parskip}{2mm} % между абзацами
\graphicspath{ {images/} }
\title{Тепловые явления}
\date{}
\author{}
\begin{document}
\section*{\centering{Тепловые явлеfasfafdния}}
\textbf{Тепловые явления}
— явления, связанные с изменением температуры или агрегатного состояния в-ва. 
\textit{(кипение в чайнике, кристаллизация в морозильнике, конденсация)} 
\subsection*{\centering{Тепловое движение}}
  \textbf{Тепловое движение} — хаотичное движение молекул, связанное с температурой.\par
Сопоставление некоторых температур, связанных с ними мест/явлений:
\begin{itemize}
\item \(0\) °C — температура плавления льда
\item \(20\)°C — средняя температура на земле
\item \(58\) °C — максимальная температура на земле
\item \(-89\) °C — минимальная температура на земле
\item \(350 — 400\) °C — Меркурий, Венера
\item \(-250\) °С — Нептун
\item \(5800\) °C поверхность солнца
\item \(15*10^6 — 17*10^6\) °C в центре солнца
\item \(200*10^6\) — получено в лаборатории
\item \(-273,15\) °С — минимально возможная температура
\end{itemize}
\end{document}
