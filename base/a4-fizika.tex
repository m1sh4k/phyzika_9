\def\PDFAuthor{m1sh4k}
\def\PdfTitle{ФИЗИКА}
\def\DocName{Конспекты по физике, 9 класс}


\def\paragraphindent{12.5mm}    % Отступ в первой строке абзаца (по горизонтали)
\def\paragraphskip{2mm}         % Отступ между абзацами (по вертикали)
\def\parleftskip{0mm}
\def\parrightskip{5mm}

\usepackage{DejaVuSansCondensed} % шрифты документа - DejaVuSansCondensed
\renewcommand*\familydefault{\sfdefault}

\usepackage{cmap} % Таблицы кодировок. Корректная работа с кириллицей (поиск, копирование)
\usepackage[T1,T2A]{fontenc} % кодовая страница
\usepackage[utf8]{inputenc} % кодировка входного файла

\usepackage[english,russian]{babel}

\usepackage[linktocpage=true]{hyperref} % Чтобы работали ссылки, pdf metadata
\hypersetup{pdfauthor   = \PDFAuthor,
            pdftitle    = \PdfTitle,
            pdfsubject  = \DocName,
            pdfkeywords = {},
            pdfproducer = {},
            pdfcreator  = {},
            bookmarks=true, % add bookmarks to index
            bookmarksopen=true, % defaults to having the bookmarks open
            unicode = true,  % проверить зечем это надо
            colorlinks = true, % Включаем цвета у ссылок (false - включается цвет рамки)
            urlcolor = blue,      % ссылки на WEB сайты
            linkcolor = black,    % внутренние ссылки
            filecolor = yellow,   % ссылки на файлы
            linkbordercolor = {1 0 0},    % цвет рамки
            pdfnewwindow = true, % открывать в новом окне
            pdfborderstyle = {/S/U/W 1}}


\usepackage{hyphenat}
\hyphenpenalty=9000 % чем выше значение (до 10000), тем меньше переносов, default=50
\uchyph=1 % не переносить слова с большой буквы (не будут переноситься имена собственные), default=1
\tolerance=10000 % точность выравнивания правого края

\usepackage{calc}

\usepackage{setspace}
\singlespacing     % 1
%\onehalfspacing    % 1.5
%\setstretch{1.25}  % 1.5
%\doublespacing      % 2
%\linespread{1.25}

\setlength{\parindent}{\paragraphindent} % абзацный отступ
\setlength{\parskip}{\paragraphskip} % между абзацами

\usepackage{graphicx}
\graphicspath{ {./images/} }
\DeclareGraphicsExtensions{.eps,.pdf,.png}

\usepackage{enumitem}
\setlist{nosep,
         leftmargin=\parleftskip,
         rightmargin=\parrightskip,
         itemindent=\paragraphindent+5mm,
         labelwidth=3mm,
         labelsep=2mm}


\usepackage{titlesec}

\titleformat
{\section} % command
[block]% shape
{\centering\bfseries\large} % format
{\hspace{\paragraphindent}\thesection\ } % label
{0pt} % sep
{} % before-code
[] % after-code
\titlespacing{\section}{\leftskip}{0em}{2em}[\rightskip]

\newcommand{\usection}[1]{\section*{#1}                     % Section  без нумерации
\addcontentsline{toc}{section}{\protect\numberline{}#1}}


\titleformat
{\subsection} % command
[block]% shape
{\centering\bfseries} % format
{\hspace{\paragraphindent}\thesubsection\ } % label
{0pt} % sep
{} % before-code
[] % after-code
\titlespacing{\subsection}{\leftskip}{2em}{2em}[\rightskip]

\titleformat
{\subsubsection} % command
[block]% shape
{\bfseries} % format
{\hspace{\paragraphindent}\thesubsubsection\ } % label
{0pt} % sep
{} % before-code
[] % after-code
\titlespacing{\subsubsection}{\leftskip}{*0}{*0}[\rightskip]



\usepackage{utfsym}     % Символы Юникод
\usepackage{xcolor}     % Управление цветом



% СТИЛЬ ВЫДЕЛЕНИЯ

\newcommand{\hlight}[2][black]{{\bfseries\color{#1}{#2}}}



